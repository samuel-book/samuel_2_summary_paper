\documentclass{article}

% Formatting
\DeclareUnicodeCharacter{2060}{\nolinebreak} % Prevent unicode (U+2060) error on local complile
\frenchspacing % No double spacing between sentences
\hbadness=1000000 % Turn off \hbox badness warnings
\linespread{1.2} % Set linespace


% Packages
\usepackage[a4paper, left=1.5cm, right=1.5cm, top=1.2cm, bottom=1.5cm]{geometry}
\usepackage{authblk} % For author formatting
\usepackage{caption} % For figure and table captions
\usepackage{cclicenses} % For creative commons license
\usepackage{float} % To force figure location after text
\usepackage{graphicx} % Adds more functionality to graphics for inclusion of figures
\usepackage{lscape} % For landscape pages
\usepackage{lineno} % Allows use of \linenumbers to add line numbers 
\usepackage{lmodern} % A scalable font - avoids erros due to non-sclabale fonts
\usepackage{longtable,booktabs}  % For tables
\usepackage{markdown} % Allow use of markdown syntax
\usepackage{microtype} % 'Improved' typesetting
\usepackage[nottoc,numbib]{tocbibind} % Add references to table of contents
\usepackage{parskip} % Adds white space between paragraphs
\usepackage{pdflscape} % To create landscape pages that show as landscape in PDF viewer
\usepackage{placeins} % to use \FloatBarrier where want a hard break between content (to force and order of text and figures
\usepackage{ragged2e} % Better right ragged edges (allows hyphenation)
\usepackage{subcaption} % Allows use of subfigures
\usepackage[super]{natbib} % Citations using superscript
\usepackage{titlesec} % For title spacing
\usepackage[toc,page]{appendix}
\usepackage{url} % Tidy web links
\usepackage[utf8]{inputenc}
\usepackage{verbatim}
\usepackage{xcolor} % For coloured text
\usepackage{xurl} % For url but with more flexible linebreaking


% Choose your own colour
\usepackage{color}
\newcommand{\mjanote}[2][\textcolor{red}{\dagger}]{\textcolor{red}{$#1$}\marginpar{\color{red}\raggedright\tiny$#1$ #2}}
\newcommand{\mjaFIXME}[1]{\textcolor{red}{[\textbf{FIXME} \textsl{#1}]}}
\newcommand{\kpnote}[2][\textcolor{magenta}{\dagger}]{\textcolor{magenta}{$#1$}\marginpar{\color{magenta}\raggedright\tiny$#1$ #2}}
\newcommand{\kpFIXME}[1]{\textcolor{magenta}{[\textbf{FIXME} \textsl{#1}]}}


% Info on wordcounts:
% https://www.overleaf.com/learn/how-to/Is_there_a_way_to_run_a_word_count_that_doesn%27t_include_LaTeX_commands%3F

% To include refs in word count:
%TC:incbib

\newcommand{\detailtexcount}[1]{%
  \immediate\write18{texcount -merge -sum -q #1.tex output.bbl > #1.wcdetail }%
  \verbatiminput{#1.wcdetail}%
}

% Count tables in wordcount

%TC:group table 0 1
%TC:group tabular 1 1

\begin{document}



\section*{\centering{SAMueL-2: Stroke Audit Machine Learning}}

\section{Abstract}

\textbf{Background}

Use of thrombolysis to treat emergency stroke patients varies considerably between hospitals. Previous work has shown that the majority of this variation comes from between-hospital decision-making on which patients should receive thrombolysis.

\textbf{Key objectives} % Probably need reducing for abstract

\begin{itemize}

    \item Use qualitative methodologies to ask \textit{"What should a machine-learning model based on national audit data look like, do, and deliver, to optimise improvement and reduce unwarranted variation, in thrombolysis?"}

    \item Use explainable machine learning to investigate what affects whether a patient receives thrombolysis or not, and what affect that thrombolysis has on patient outcome. Use \textit{prototype patients} to further illustrate the relationship between patient features, expected use of thrombolysis, and expected patient outcomes.
    
    \item Incorporate health economics in patient outcome modelling.

    \item Model patient flow from stroke onset to disability at discharge. 

\end{itemize}


\textbf{Methods}

\begin{itemize}

    \item \textbf{Co-production and qualitative research} used observations, meetings, semi-structured interviews, and review of NHS documentation.

    \item \textbf{Thrombolysis decisions and patient outcomes} were predicted using XGBoost machine learning models, employing Shapley values to explain individual feature contributions.

    \item Scenarios of \textbf{patient flow through the emergency stroke pathway} was modelled with Monte-Carlo simulation, and a mathematical model of outcome based on clinical trial results.
    
\end{itemize}

\textbf{Results}

\begin{itemize}

    \item \textbf{Qualitative insights} found optimism among participants about using SAMueL-2 technology to standardise thrombolysis practices, being especially beneficial for less experienced clinicians for training and case review. It identified the importance to reassure adopters about the integrity of modelling based on SSNAP, and that emergency department physicians have less confidence about the evidence base for thrombolysis.
    
    \item \textbf{Key factors affecting thrombolysis use} in ischaemic stroke are arrival-to-scan time, stroke severity, pre-stroke disability, and the hospital attended. After adjusting for other patient features, a 13-fold variability in thrombolysis odds was observed between-hospitals.
    
    \item \textbf{Key factors affecting outcome} after stroke are pre-stroke disability, stroke severity, age, and use/time of thrombolysis. Machine learning demonstrated that thrombolysis was having at least the expected clinical benefit as predicted by the clinical trials. Thrombolysis was predicted to add an average of 0.26 QALYs per patient treated. Hospitals with higher thrombolysis use are predicted to be generating better patient outcomes.
    
    \item Combining \textbf{changes to processes and decision-making} could increase thrombolysis use from 13\% to 20\% among patients arriving by ambulance. Accelerating pathway speed positively influenced outcomes more than thrombolysis rates. A web tool was developed for stroke team-level data interrogation.

    \end{itemize}

\textbf{Conclusions}

Using large-scale observational data and machine learning, thrombolysis, in real world use, was found to have at least as much benefit as predicted by the thrombolysis clinical trial meta-analysis. Both qualitative research and machine learning revealed significant between-hospital variation in which patients receive thrombolysis, which is leading to significant between-hospital variation in thrombolysis use and outcomes. Machine learning revealed that who will benefit from thrombolysis is patient-specific, and not easily captured in a simple medicine use label, but we found overall that stroke teams with a higher willingness to use thrombolysis are predicted to be generating better patient outcomes at a population level.

\section{Overview of associated papers}

\begin{landscape}
\RaggedRight
\small
\renewcommand{\arraystretch}{1.2}
\begin{longtable}{p{6cm}|p{6cm}|p{8cm}|p{5cm}}
\caption{Summary of five SAMueL-2 papers}\label{tab:papers}\\
%\toprule
Paper & Study type and population & Key findings & Notes/resources\tabularnewline
\midrule
\endhead

%%%%%%%%%%%%%%%%%%%%  Paper 1: Qualitative
Stroke physicians' and staff perspectives on machine learning to optimise thrombolysis decision making in stroke: a qualitative study. \cite{jarvie_stroke_2024} &

Qualitative study: 184 hours of focussed observation in three NHS Trusts in England. Observation of Integrated Stroke Delivery Networks (ISDNs) and other organisations with strategic overview of stroke services. 20 participants from the three observation sites and five key informants from other sites took part in semi-structured interviews.&

* Participants were hopeful the SAMueL-2 technology could address variance in thrombolysis practice. It was seen as particularly suitable for junior clinicians, non-stroke specialists and at district general hospitals and offered value for training, reviewing clinical cases, and quality improvement.

\vspace{2mm}

* It is important to reassure intended adopters about the integrity of modelling based on this data.

\vspace{2mm}

* Evidence indicated ED physicians may have less confidence in the evidence base for thrombolysis.

\vspace{2mm}

* Perceived lack of funding and stroke workforce shortages may impede quality improvement and adoption of new technologies such as SAMueL-2.
& \tabularnewline

\midrule
%%%%%%%%%%%%%%%%%%%%%%%% % Paper 2
What Would Other Emergency Stroke Teams Do? Using Explainable Machine Learning to Understand Variation in Thrombolysis Practice.\cite{pearn_what_2023}&

Explainable machine learning model to study decision-to-treat with thrombolysis across 132 stroke teams.  

\vspace{2mm}

Model based on 3 years of all emergency stroke arrivals to hospitals in England and Wales, arriving within 4 hours of stroke onset (88,928 patients). & 


* Thrombolysis use in patients arriving within 4 h of known or estimated stroke onset ranged 7\% -49\% between hospitals. 

\vspace{2mm}

* The odds of receiving thrombolysis reduced 9-fold over the first 120 min of arrival-to-scan time, varied 30- fold with stroke severity, reduced 3-fold with estimated rather than precise stroke onset time, fell 6-fold with increasing pre-stroke disability, fell 4-fold with onset during sleep, fell 5-fold with use of anticoagulants, fell 2-fold between 80 and 110 years of age, reduced 3-fold between 120 and 240 min of onset-to-arrival time and varied 13-fold between hospitals. 

\vspace{2mm}

* The majority of between-hospital variance was explained by the hospital, rather than the differences in local patient populations. & 

GitHub pages book:

\url{https://samuel-book.github.io/samuel_shap_paper_1/}

\vspace{2mm}

GitHub code and full results:

\url{https://github.com/samuel-book/samuel_shap_paper_1}

\tabularnewline

\midrule
%%%%%%%%%%%%%%%%%%%%%%%%%%%% Paper 3
Thrombolysis: Are the results from the clinical trial meta-analysis seen in real life outcomes?  A machine learning study of the UK stroke registry.\cite{pearn_are_2024} &

Explainable machine learning model to investigate clinical benefit of thrombolysis (in those patients who received thrombolysis), compared with clinical trials.

\vspace{2mm}

Model based on 6 years of all emergency ischaemic stroke arrivals to hospitals in England and Wales (168,347 patients).& 

* Thrombolysis was found to be associated with a statistically significant improvement in the odds of having a good outcome using any mRS threshold. Regression analysis predicted a maximum 2.5-fold improvement in odds of achieving mRS 0-1, with a decline to no treatment effect at 5 hours 28 minutes post-onset.

\vspace{2mm}

* The observed beneficial effect is very similar to Emberson’s meta-analysis of a maximum 2.0-fold improvement in odds of achieving mRS 0-1, with a decline to no treatment effect at 6 hours 88 minutes post-onset \cite{emberson_effect_2014}. & 

GitHub code and full results: 

\url{https://github.com/samuel-book/thrombolysis_clinical_trials_ml_paper}
\tabularnewline

\midrule
%%%%%%%%%%%%%%%%%%%%%%%%%%%%%%%%%%%%%%%%% Paper 4
Are the patients who would benefit from thrombolysis the same ones as those receiving it? A machine learning study of the UK stroke registry.\cite{pearn_are_2024} &

Explainable machine learning model to investigate thrombolysis decision making (at stroke unit level) and clinical benefit of thrombolysis.

\vspace{2mm}

Model based on 6 years of emergency ischaemic stroke admissions that that arrived at an acute stroke team in England and Wales by ambulance after their stroke onset,  had their scan within 255 minutes of stroke onset, and were not receiving anticoagulants for atrial fibrillation ( 91,464 patients).&

* 45\% of the study population received thrombolysis.

\vspace{2mm}

* 60\% of the study population were predicted to have a better outcome with thrombolysis (improved probability-weighted mRS and reduced probability of mRS 5-6).

\vspace{2mm}

* 73\% of those treated were predicted to have a better outcome with thrombolysis, and 49\% of those not treated were predicted to have a better outcome with thrombolysis. 

\vspace{2mm}

* Patients with mismatched treatment decisions (actual thrombolysis use vs. predicted to benefit) can not be identified from an isolated feature value. Individual hospitals vary in balancing maximising benefit from thrombolysis vs. avoiding any possible harm.&

GitHub code and full results:

\url{https://github.com/samuel-book/stroke_outcome}
\tabularnewline

\midrule
% %%%%%%%%%%%%%%%%%%%%%%%%%%%Paper 5
Identifying levers for improving thrombolysis use and outcomes – combining clinical pathway simulation and machine learning applied to the UK stroke registry.\cite{pearn_identifying_2024}&

Clinical pathway simulation, mathematical outcome modelling, and machine learning to investigate thrombolysis decision making and outcomes after stroke.

\vspace{2mm}

Model based on 5 years of all emergency stroke arrivals, by ambulance, to hospitals in England and Wales (302,715 patients).& 

* Combining potential changes (improving arrival-to-thrombolysis times to 30 mins, reducing ambulance call-to-hospital-arrival times by 15 minutes, having all teams attain the current upper-quartile performance in determining stroke onset time, and applying decision making similar to high thrombolysing units) would be expected to increase thrombolysis use in patients arriving by ambulance from 13\% to 20\% and double the clinical benefit (additional patients discharged mRS 0-1) from thrombolysis.


\vspace{2mm}
 
* The largest single factor in improving both thrombolysis use and outcomes is differences in clinical decision-making.

\vspace{2mm}

* High thrombolysing units are predicted to be generating more net benefit (including better avoidance of mortality and severe disability) than low thrombolylsing units.


\vspace{2mm}

* A health economics model, using the output from the outcome prediction model, estimated that thrombolysis produces an additional 0.26 QALY for each person treated.&

GitHub code with dummy data:

\url{https://github.com/samuel-book/samuel_2_demo}

\vspace{2mm}

Web application:

\url{https://stroke-predictions.streamlit.app/}


\tabularnewline
\bottomrule
\end{longtable}
\normalsize
\end{landscape}

\bibliographystyle{naturemag}
\bibliography{references}


\end{document}
