\section*{Abstract} %* makes it not numbered
\addcontentsline{toc}{section}{Abstract} %Add non-numbered section to table of contents

\textbf{Background}

Use of thrombolysis to treat emergency stroke patients varies considerably between hospitals. Previous work has shown that the majority of this variation comes from between-hospital decision-making on which patients should receive thrombolysis.

\textbf{Key objectives}

\begin{itemize}

    \item Generate empirically and theoretically informed knowledge about how thrombolysis is currently delivered, centred on physicians’ views, understandings, and practices.
    
    \item Learn how stroke physicians’ and staff think and feel about the use of machine learning based on SSNAP in improving clinical practice, and identify barriers for it's use.

    \item Understand how patient characteristics affect decisions to use, or not use, thrombolysis at each hospital.

    \item Predict expected outcomes for patients, with and without thrombolysis, and investigate whether hospitals with higher use of thrombolysis are likely to be generating more patient benefit. 
    
    \item Model patient flow (including ambulance response times) and predict how process changes would affect thrombolysis use and benefit.

    \item Predict how thrombolysis affects quality-adjusted life years (QALYs).

    \item Investigate how well organisation factors predict thrombolysis use, to identify whether aspects of the organisation impacts the decision to treat.
    
    \item Co-production of project outputs with clinicians to promote acceptance and use for local quality improvement.

    \item Generation of a web-based dashboard to allow clinicians to interrogate modelling.

\end{itemize}

\textbf{Methods}

\begin{itemize}

    \item Co-production and qualitative research used observation, semi-structured interviews, and review of NHS documents.

    \item  Thrombolysis decisions and outcomes (with and without thrombolysis) were predicted using XGBoost machine learning, with Shapley values  generated to show the contribution of individual features to the prediction.

    \item Patient flow through the emergency stroke pathway was modelled with Monte-Carlo simulation, and a mathematical model of outcome based on  clinical trial results.
    
\end{itemize}

\textbf{Results}

\begin{itemize}

    \item Qualitative research showed that participants were hopeful the SAMueL-technology could address variance in thrombolysis practice. It was seen as particularly suitable for less experienced clinicians and offered value for training, reviewing clinical cases, and quality improvement. It is important to reassure intended adopters about the integrity of modelling based on this data. Evidence indicated that emergency department physicians may have less confidence in the evidence base for thrombolysis.
    
    \item Factors that most affect thrombolysis use in ischaemic stroke are arrival-to-scan time, stroke severity, prior disability, and hospital attended. After adjusting for other patients features, the odds of receiving thrombolysis varied 13-fold between hospitals.

    \item Factors that most affect outcome after stroke are pre-stroke disability, stroke severity, age and use/time of thrombolysis.
    
    \item Machine learning demonstrated that thrombolysis was having at least the expected clinical benefit as predicted by the clinical trials. On average, it was predicted that thrombolysis adds 0.26 QALY for each person treated. Stroke teams with higher thrombolysis use are predicted to be generating better patient outcomes.

    \item By combining changes to processes and decision-making there is potential to increase thrombolysis use from 13\% to 20\% in patients arriving by ambulance. Improving pathway speed has a greater effect on outcomes than on thrombolysis use. A web tool allows interrogation at the level of stroke team.
\end{itemize}

\textbf{Conclusions}

We combined qualitative research and large-scale observational data with machine learning on use of, and outcomes from, thrombolysis. Both qualitative research and machine learning revealed significant between-hospital variation in which patients receive thrombolysis. Machine learning revealed that who will benefit from thrombolysis is patient-specific, and not easily captured in a simple medicine use label, but we found overall that stroke teams with a higher willingness to use thrombolysis are predicted to be generating better patient outcomes at a population level.

\section*{Plain Language Summary}
\addcontentsline{toc}{section}{Plain Language Summary}

\textbf{What is the problem?} Use of clot-busting treatment (`\textit{thrombolysis}') in stroke varies a great deal between hospitals.

\textbf{What did we know?} We know that the largest cause of this variation is in how doctors decide which patients are suitable for thrombolysis. Some clinicians are worried that the risks can outweigh the benefits, and that use of this treatment in the real world won’t have the same benefits that were predicted by the clinical trials.

\textbf{What did we not know?} We did not know (1) how this variation in use of thrombolysis was affecting patient outcomes, (2) what it was about the patients that doctors considered when making decisions, (3) what most affected patient outcome, (4) what doctors would think of using artificial intelligence (`AI') to help understand answers to these questions.

\textbf{What did we do?} We used AI to learn which patients different hospitals would give thrombolysis to, and to learn which patients would likely have a better outcome if this treatment was used.

\textbf{What did we find?} We found that thrombolysis was at least as effective in real-world use as the original clinical trials predicted, and that hospitals choosing to use it more are very likely saving more lives and reducing disability from stroke. We now understand what factors influence doctors the most when deciding to use it or not, and what most affects patient outcomes. We found that speeding up thrombolysis would not increase the number of people who receive it much, but it would mean everyone receiving it would benefit more. We found that clinicians felt that our work could help improve their practice, but they needed more convincing that our ‘AI’ methods were valid. We are hopeful that our results will give clinicians greater confidence to use this treatment more often and in more patients, and to always give it as fast as possible.