\subsection{Data}

\subsection{Data for modelling}

Data extracted from the national stroke registry for England, Wales and Northern Ireland, the Sentinel Stroke National Audit Programme (SSNAP) for all patients with out-of-hospital stroke for the full calendar years of 2016-2021. The registry contains all consecutive patients admitted to 100\% of acutely-admitting hospitals with a case ascertainment of over 90\% when compared to administrative data (Hospital Episode Statistics). The total number of patients was 360,381, of whom 38.5\% arrived within 4 hours of known stroke onset. Of those arriving within 4 hours of known stroke onset 90.4\% arrived by ambulance (71.6\% of those not arriving within 4 hours of known stroke onset arrived by ambulance).

The following data fields from SSNAP were used in the modelling:

\begin{itemize}

    \item \textit{Stroke team}: Stroke team attended (hospital identifier).

    \item \textit{Age}: As midpoint of 5 year age bands.

    \item \textit{Sex}: Sex of patient (male/female)

    \item \textit{Diagnosis of atrial fibrillation}: Did the patient have a diagnosis of atrial fibrillation, either made prior to admission, or during admission?

    \item \textit{Use of anticoagulants}: Use of prior anticoagulant for atrial fibrillation.

    \item \textit{Onset known}: Whether onset was known, and if known whether it was considered to be known precisely or was a best estimate.

    \item \textit{Onset during sleep}: Did stroke occur in sleep? (1 = Yes, 0 = No).

    \item \textit{Onset-to-arrival time}: Time from onset of stroke to arrival at hospital (minutes), when known.

    \item \textit{Prior disability level}: Estimated modified Rankin Scale, mRS, prior to stroke.

    \item \textit{Stroke type}: Infarction/haemorrhage.

    \item \textit{Stroke severity}: National Institutes of Health Stroke Scale (NIHSS) score on arrival.

    \item \textit{Arrival-to-scan time}: Time from arrival at hospital to scan (minutes), when known.

    \item \textit{Scan-to-thrombolysis time}: Time from arrival at hospital to scan to treatment with thrombolysis  (minutes), when given.

    \item \textit{Disability on discharge}: mRS (0-6) on discharge, includes death (mRS 6) during admission.
    
\end{itemize}

\subsubsection{Ethics}

For modelling, as we were using anonymised secondary data, collected for national audit, individual consent is not required. SSNAP has approval under section 251 of the NHS Health and Social Care Act (2006) to collect patient level data on the first six months of patient care (ECC 6- 02(FT3)/2012), without requiring individual patient consent. Access to SSNAP data is managed by the UK Healthcare Quality Improvement Partnership (HQIP), with this project being approved by HQIP (HQIP303). More information on SSNAPs use of patient data may be found at: \url{https://www.strokeaudit.org/ SupportFiles/Documents/Patient-area-documents/Fair-processingstatement-for-patients-v7-0.aspx}

As we are using anonymised secondary data, collected for national audit, used for service evaluation and improvement, no ethical approval is required (confirmed using the NHS Health Research Authority decision aid: https://www.hra-decisiontools.org.uk/ethics/).

\subsection{Qualitative data}

Qualitative research used a combination of observations, interviews and documentary analysis.

Semi-structured interviews were conducted with 20 participants from the three observation sites and five key informants (senior clinicians/managers involved in stroke initiatives) based at other sites. The number of participants by role was:

\begin{itemize}

    \item \textit{Consultant}: 11 Stroke, 3 Emergency Department (ED), 1 Care of the Elderly

    \item \textit{Associate Specialist Doctor (ED)}: 1 ED

    \item \textit{Registrar}: 1 Stroke, 1 ED 

    \item \textit{Stroke Assessor \& Nurse}: 2

    \item \textit{Stroke Nurse}: 1
    
    \item \textit{Advanced Clinical Practitioner (Stroke)}: 1

    \item \textit{Stroke Unit Administrator}: 1

    \item \textit{IT Co-ordinator (Stroke)}: 1

    \item \textit{Integrated Stroke Delivery Network manager}: 1
    
\end{itemize}

Observation of healthcare professionals and work practices involved in the acute stroke pathway in three NHS Trusts (acute hospitals) in England (hereafter referred to as Sites A, B and C) was conducted. We used SSNAP data to purposefully select hospitals with low rates of thrombolysis and ensured they had differing stroke pathways and were geographically dispersed. 184 hours of focussed observation across the three sites comprising a variety of day/evening/night and weekend shifts was conducted, observing stroke care as well as clinical governance and multi-disciplinary team stroke review and SSNAP review meetings. Additionally, observation of online meetings of Integrated Stroke Delivery Networks (ISDNs) and other organisations with strategic overview of stroke services in the NHS took place, with a focus on thrombolysis. 

A third source of data was documents: we collected Trust-level documents (such as thrombolysis protocols and quality improvement (QI) initiative documentation); policy and strategy documents at ISDN level; computer modellers’ presentations to stakeholders (clinicians, senior managers, policymakers); modelling team summaries of feedback from stakeholders.

\subsubsection{Ethics}

HRA \& HCRW approval was provided by the Essex Research Ethics Committee in August 2023 (IRAS 322303; REC reference 23/EE/0124).