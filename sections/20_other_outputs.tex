\section{Other project outputs}

\subsection{Production and demonstration code, with artificial patient data}

Core analysis from this project is being implemented by SSNAP. This will give each team a bespoke thrombolysis target, based on theor own local patient populations. For this, code, with artificial patient data, has been made available at \url{https://github.com/samuel-book/samuel_2_demo}. Artificial data made available is produced by sampling patient attributes separately from distributions based on real patient data attending each stroke team (with rounding/censoring process times and ages); covariances between sampled features are not maintained. This method may sometimes combine combinations of patient features that are not realistic of real patients. For machine learning data, the use of thrombolysis and the outcome data was generated by machine learning models predicting use of thrombolysis and outcome for that individual artificial patient. This data allows demonstration of the models, maintain many of the relationships between patient characteristics and use/outcome of thrombolysis, but should never be used for clinical research into stroke or for guidance on clinical decision-making.

\subsection{SAMueL-2 Web Application}

A web application has been made available to stroke teams at \url{https://stroke-predictions.streamlit.app/}. This web application allows teams to see descriptive statistics of all stroke teams, an analysis of the effect of potential pathway changes at each stroke team (using anonymised stroke team identity), and what thrombolysis decision each stroke team would make on a customisable patient (using anonymised stroke team identity)/

\subsection{Thrombolysis in Acute Stroke Collaborative (TASC)}

The SAMueL team has been providing bespoke reports to the \textit{Thrombolysis in Acute Stroke Collaborative} (TASC). TASC is an NHS-England sponsored and funded project working with NHS Elect. TASC is taking 6 low thrombolysing stroke teams and is helping them improve thrombolysis use. A follow up project, focussing on 12-15 more teams has been agreed.

\subsection{Stroke unit demographics}

To provide background on stroke units and their population demographics we have combined data from multiple sources (at Lower Super Output Area) and collated according to stroke unit catchment areas (estimated by assigning each LSOA to the stroke team with the shortest travel time).

Data and code for this is available at: url{https://github.com/samuel-book/stroke_unit_demographics}. Analysis is also made avilable by a web application at: \url{https://stroke-unit-demographics.streamlit.app/Interactive_demo}.

\subsection{Predicting willingness to use thrombolysis from SSNAP organisational audit data}

We performed a model to investigate whether organisational audit results reported by SSNAP could be used to predict a teams willingness to use thrombolysis (as measured by the SHAP value for the stroke team in the thrombolysis decision model). We were not able to predict willingness to use thrombolysis from these data (see \url{https://github.com/samuel-book/thrombolysis_organisational_factors/blob/main/data_prep/2c_statsmodels.ipynb}).