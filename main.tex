\documentclass{article}

% Formatting
\DeclareUnicodeCharacter{2060}{\nolinebreak} % Prevent unicode (U+2060) error on local complile
\frenchspacing % No double spacing between sentences
\hbadness=1000000 % Turn off \hbox badness warnings
\linespread{1.2} % Set linespace


% Packages
\usepackage[a4paper, left=2.5cm, right=2.5cm, top=2.5cm, bottom=2.5cm]{geometry}
\usepackage{authblk} % For author formatting
\usepackage{caption} % For figure and table captions
\usepackage{cclicenses} % For creative commons license
\usepackage{float} % To force figure location after text
\usepackage{graphicx} % Adds more functionality to graphics for inclusion of figures
\usepackage{lscape} % For landscape pages
\usepackage{lineno} % Allows use of \linenumbers to add line numbers 
\usepackage{lmodern} % A scalable font - avoids erros due to non-sclabale fonts
\usepackage{longtable,booktabs}  % For tables
\usepackage{microtype} % 'Improved' typesetting
\usepackage[nottoc,numbib]{tocbibind} % Add references to table of contents
\usepackage{parskip} % Adds white space between paragraphs
\usepackage{pdflscape} % To create landscape pages that show as landscape in PDF viewer
\usepackage{ragged2e} % Better right ragged edges (allows hyphenation)
\usepackage{subcaption} % Allows use of subfigures
\usepackage[super]{natbib} % Citations using superscript
\usepackage{titlesec} % For title spacing
\usepackage[toc,page]{appendix}
\usepackage{url} % Tidy web links
\usepackage[utf8]{inputenc}
\usepackage{verbatim}
\usepackage{xcolor} % For coloured text
\usepackage{xurl} % For url but with more flexible linebreaking


% Choose your own colour
\usepackage{color}
\newcommand{\mjanote}[2][\textcolor{red}{\dagger}]{\textcolor{red}{$#1$}\marginpar{\color{red}\raggedright\tiny$#1$ #2}}
\newcommand{\mjaFIXME}[1]{\textcolor{red}{[\textbf{FIXME} \textsl{#1}]}}
\newcommand{\kpnote}[2][\textcolor{magenta}{\dagger}]{\textcolor{magenta}{$#1$}\marginpar{\color{magenta}\raggedright\tiny$#1$ #2}}
\newcommand{\kpFIXME}[1]{\textcolor{magenta}{[\textbf{FIXME} \textsl{#1}]}}


% Info on wordcounts:
% https://www.overleaf.com/learn/how-to/Is_there_a_way_to_run_a_word_count_that_doesn%27t_include_LaTeX_commands%3F

% To include refs in word count:
%TC:incbib


\newcommand{\detailtexcount}[1]{%
  \immediate\write18{texcount -merge -sum -q #1.tex output.bbl > #1.wcdetail }%
  \verbatiminput{#1.wcdetail}%
}

\newcommand{\quickwordcount}[1]{%
  \immediate\write18{texcount -1 -sum -merge -q #1.tex output.bbl > #1-words.sum }%
  \input{#1-words.sum} words%
}

\newcommand{\quickcharcount}[1]{%
  \immediate\write18{texcount -1 -sum -merge -char -q #1.tex output.bbl > #1-chars.sum }%
  \input{#1-chars.sum} characters (not including spaces)%
}


% Count tables in wordcount

%TC:group table 0 1
%TC:group tabular 1 1

\begin{document}

% Ignore title and abstract in word count
%TC:ignore
\title{Title}


\renewcommand{\thefootnote}{\fnsymbol{footnote}}
\author[1,2]{Kerry Pearn}
\author[*1,2]{Michael Allen}
\author[1,2]{Martin James}

% Check affiliations - update RDE Name
\affil[1]{\footnotesize University of Exeter Medical School}
\affil[2]{\footnotesize NIHR South West Peninsula Applied Research Collaboration (ARC).}
\affil[*]{\footnotesize Corresponding author: m.allen@exeter.ac.uk}

\maketitle
%TC:endignore

\newpage

Test text

\begin{landscape}
{
\RaggedRight
\begin{table}
\small
\raggedright
\begin{tabular}{|p{5cm}|p{5cm}|p{6cm}|p{6cm}|}

\hline
\textbf{Model} &  \textbf{Type} & \textbf{Required data} & \textbf{Outputs} \\
\hline

% Decision model
\textbf{Thrombolysis decision model} 

\vspace{2mm}

Learns which patients will receive thrombolysis at each stroke team. & 

XGBoost machine learning model + SHAP model for explianability. & 

Age\newline\vspace{3pt}
Arrival-to-scan time\newline\vspace{3pt}
Onset during sleep (Y/N)\newline\vspace{3pt}
Onset time type (precise/estimated)\newline\vspace{3pt}
Onset-to-arrival time\newline\vspace{3pt}
Pre-stroke disability (mRS)\newline\vspace{3pt}
Stroke severity\newline\vspace{3pt}
Stroke team\newline\vspace{3pt}
Stroke type\newline\vspace{3pt}
Use of anticoagulants for atrial fibrillation & 

Probability of receiving thrombolysis at attended stroke team and 25 \textit{benchmark} stroke teams. \\

\hline

% Outcome model

\textbf{Stroke outcome model}

\vspace{3mm}

Learns death/disability outcome depending on patient characteristics, stroke team attended, and use/time of thrombolysis. &

XGBoost machine learning model + SHAP model for explainability. &

Age\newline\vspace{3pt}
Diagnosis of atrial fibrillation\newline\vspace{3pt}
Onset time type (precise/estimated)\newline\vspace{3pt}
Onset-to-thrombolysis time\newline\vspace{3pt}
Pre-stroke disability (mRS)\newline\vspace{3pt}
Stroke severity\newline\vspace{3pt}
Stroke team\newline &

Disability/death distribution (can be estimated with/without thrombolysis for any patient). \\

\hline

% Pathway model

\textbf{Pathway model}

\vspace{3mm}

Models flow patients through each stroke team’s thrombolysis pathway, and predicts thrombolysis use and outcomes. &

Discrete event simulation (using NumPy) array, coupled with clinical outcome model based on time to thrombolysis. &

Arrival-to-scan time\newline\vspace{3pt}
Arrive by ambulance (Y/N)\newline\vspace{3pt}
Onset known (Y/N)\newline\vspace{3pt}
Onset-to-arrival time\newline\vspace{3pt}
Scan-to-thrombolysis time\newline\vspace{3pt}
Stroke team\newline\vspace{3pt}
Stroke type &

Use of thrombolsyis and number of people mRS 0-1 under various improvement strategies:\newline\vspace{6pt}

1. Current state\newline\vspace{3pt}
2. Ascertain stroke onset times at performance of 25\textsuperscript{th} percentile stroke team\newline\vspace{3pt}
3. 30 minutes arrival-to-thrombolysis\newline\vspace{3pt}
4. Make decisions inline with \textit{benchmark} stroke teams\newline\vspace{3pt}
5. Combinations of above\newline\vspace{3pt}
\\

\hline


\end{tabular}
\caption{SAMueL model components overview}
\label{tab:example}
\end{table}
}
\end{landscape}

Test text

% Word counts - Don't count these!
%TC:ignore
\section{Word counts}
\detailtexcount{main}
%TC:endignore


\end{document}